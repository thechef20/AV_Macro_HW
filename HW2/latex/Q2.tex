\begin{enumerate}
    \item  
    $$ c^t_h(t+1) = w(t)h^h_t(t)R(t)+(1+n-R(t))*T-R(t)c^h_t$$
    $$\max_{c^h_t(t)} ~ ln(c_t^h)+\beta ln(w(t)h^h_t(t)R(t)+(1+n-R(t))*T-R(t)c^h_t)$$
    take the first derivative 
    $$ \frac{1}{c_t^h}+\beta*\frac{-R(t)}{(w(t)h^h_t(t)R(t)+(1+n-R(t))*T-R(t)c^h_t)}=0$$
    $$ \frac{1}{c_t^h}=\beta*\frac{R(t)}{(w(t)h^h_t(t)R(t)+(1+n-R(t))*T-R(t)c^h_t)}$$
    $$ \beta R(t)c_t^h=(w(t)h^h_t(t)R(t)+(1+n-R(t))*T-R(t)c^h_t)$$
    $$ \beta R(t)c_t^h+R(t)c^h_t=w(t)h^h_t(t)R(t)+(1+n-R(t))*T$$
    $$ (R(t)+R(t)\beta)c^h_t=w(t)h^h_t(t)R(t)+(1+n-R(t))*T$$
    $$ c^h_t=\frac{w(t)h^h_t(t)R(t)+(1+n-R(t))*T}{(R(t)+R(t)\beta)}$$
    now we will look at the aggregated case
    $$ C_t=\frac{w(t)H^h_t(t)R(t)+(1+n-R(t))*T}{R(t)(1+\beta)}$$
    $$ C_t=\frac{1}{R(t)(1+\beta)}((1-\theta)K(t)^{\theta} H(t)^{\theta-1}R(t)+(1+n-R(t))*T)$$
    $$ C_t=\frac{1}{R(t)(1+\beta)}((1-\theta)Y(t)R(t)+(1+n-R(t))*T)$$
    $$ C_t=\frac{1}{(1+\beta)}((1-\theta)Y(t)+\frac{(1+n)*T}{R(t)}-T)$$
    
    \item
    if $(1+n)>R(t)$ aggregate consumption of the young will be higher with pay as you go because the population is growing faster than the interest rate thus the new young will be producing output that will make it advantageous for the system to use the pay as you go. (this is a problem with the baby boomer generation) 
    \item
    Budget constrain when Young: $w(t)h^h_t=c^h_t(t)+l^h(t)+k^h(t+1)+T$\\
     Budget constrain when Young: $w(t)h^h_t=c^h_t(t)+l^h(t)+k^h(t+1)+T$\\
     Total budget constraint $C^h_t(t)+\frac{c^h_t(t+1)}{R(t)}= w(t)h_t^h+\frac{R(t)-R(t)}{R(t)}T$\\= $C^h_t(t)+\frac{c^h_t(t+1)}{R(t)}= w(t)h_t^h$\\
     for all intents in proposes the tax acts as a forced investment vehicle. As such the consumption by the young will essentially be the same!
     

\end{enumerate}